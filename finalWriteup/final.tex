\documentclass{scrartcl}
\usepackage[margin=1in]{geometry}
\usepackage{url}

\title{Jurassic Parkour}
\subtitle{Reinforcement Learning in an HTML5 Browser Game}
\author{Sebastian Laudenschlager, Ian Martiny}

\begin{document}
\maketitle

\section{Introduction}

The goal of this project is to use reinforcement learning to train a computer agent to play the Chrome T-Rex game. To that end, we use data that is procedurally generated by running the game.

\section{Related Work}
Our work draws from varying motivations and uses multiple resources we'd like to
acknowledge here. The two course provided sources \cite{rltutorial, rlblog} were
very helpful in learning the information necessary to use any reinforcement
learning techniques.

We examined many different tools to use for our project. Initially we attempted
to look at PyBrain~\cite{pybrain}, a \texttt{Python} library for reinforcement
learning. While promising and accomplished it we ultimately ran into issues
attempting to do our learning on a \texttt{Python} server, while the game ran on
a \texttt{JavaScript} client. Our original plan was to run the client and pass
a score to a \texttt{Python} server which then uses a reinforcement learning
library to compute new parameters. However with the nature of our game needing
to update the ``agent'' with a procedurally generated environment required a
much quicker and constant interaction between the game and the agent.

To this end we examined other projects which were based in \texttt{JavaScript}
that used reinforcement learning. One in particular that we found useful was
FlappyBirdRL~\cite{flappybirdrl}. Though it appears that this project has a self
designed reinforcement learning library tailored specifically to their project.
While interesting we were unable to adapt this to fit our needs.

Finally we examined ReinforceJS~\cite{reinforcejs}, a \texttt{JavaScript}
library for reinforcement learning. This fit our needs perfectly: a
self-contained library, meant to interact with \texttt{JavaScript} programs
directly.
\nocite{rlblogex}

\section{Methods}

\section{Results}

\section{Conclusion}



\bibliographystyle{unsrt}
\bibliography{sources}
\end{document}
