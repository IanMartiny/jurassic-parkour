\documentclass{scrartcl}
\usepackage[margin=1in]{geometry}

\title{Jurassic Parkour}
\subtitle{Reinforcement Learning in an HTML5 Browser Game}
\author{Sebastian Laudenschlager, Ian Martiny}

\begin{document}
\maketitle

\section{The data}

Our data is unusual in that we do not have a corpus of data, but rather it is
``generated'' by the playing of the Chrome T-rex game. Additionally, given that
we are investigating reinforcement learning we do not utilize much information
from the game. We only inspect a final score.

The game in and of itself gives the player a score based on the ``distance''
that the T-rex runs. While our goal is to maximize the distance that the T-rex
runs we wish to impose additional constraints. Namely, we want our model to
learn that there are ``correct'' times (distances) to jump and that the T-rex 
should not be constantly jumping. In order to meet these goals we modify the
score given by the game to reduce points based on extraneous jumps (more than
necessary to clear an obstacle) and award points for jumping within an
acceptable distance from an obstacle and deduct points for jumping outside that
range.

\section{Baselines}

Our approach has two baselines: choosing a random threshold for the T-rex to
jump, independent of score, and starting with a low threshold for the T-rex to
jump and slowly increasing it based on score.

Both models required us to adapt the Javascript for the T-rex game to
communicate with a locally hosted Python server (Flask). The Javascript was
modified to check, at every 100th update, a random value against a threshold
that was computed by the Python server in between runs.

We found that checking the random value against the threshold at every update
caused the T-rex to be constantly jumping (even with a very high threshold), due
to the T-rex ``updating'' extremely often. 

The Javascript ``updates'' the game for every object draw and distance moved.

\subsection{Random jumping}
For the random jumping baseline a random threshold (\texttt{double} between 0
and 1) is computed by the Python server and sent to the Javascript client. The
client then begins a new game, and during every 100th update the client computes
its own random \texttt{double} between 0 and 1. The client then compares these
values, if the client's value is larger, the T-rex jumps, otherwise nothing
happens. Once the T-rex dies (by hitting an object) the client sends the score
(distance run) to the server. The server then ignores this data and computes a
new random threshold, which it sends to the client, the process then repeats.

\subsection{Increasing Threshold}
Our second baseline is to steadily increase the threshold for the T-rex to jump.
The idea is that being more discerning about when to jump will benefit the
T-rex. The increase in threshold works as follows:
\begin{itemize}
  \item If the reported score is between 1000 and 2000 (inclusive) then the new
    threshold is increased by a tenth of the distance from 1. These are
    considered low scores. A score less than 1000 is not possible. This allows
    the threshold to increase by a significant amount, without going above 1.
  \item If the reported score is between 2001 and 4000 (inclusive) then the new
    threshold is increased by a fiftieth of the distance from 1. These are still
    relatively bad scores, but realistically the best we can hope for with
    purely random jumping. For this model these scores are considered good
    enough to not change much.
  \item If the reported score is anything higher the threshold is increased by a
    hundredth of the distance from 1. Any score above 4000 is considered very
    good for this model and thus the threshold should not change very much.
\end{itemize}


\end{document}
