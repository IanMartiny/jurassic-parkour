\documentclass{scrartcl}
\usepackage[margin=1in]{geometry}

\title{Jurassic Parkour}
\subtitle{Reinforcement Learning in an HTML5 Browser Game}
\author{Sebastian Laudenschlager, Nicholas Lewchenko, Ian Martiny}

\begin{document}
\maketitle

\section{The project}

\subsection{Task}
We are teaching an automatic player for Google Chrome's jumping
dinosaur game.  The game involves reacting to obstacles by jumping
over or ducking under them.  Our automatic player must learn how
jumping and ducking in relation to obstacles affects the score over a
series of games.  We can provide further learning goals by augmenting
the scoring system to impose costs on certain actions.

\subsection{Data}
Our data is not in a corpus, but rather is procedurally generated by
the game environment, and is tagged by the in-game score provided as
well as our augmentations (number of jumps, time spent ducking, total
number of inputs, etc.).

\subsection{Baselines}
An initial baseline will be a server-client system in which the server
produces a random set of numbers and then instantiates a client game
session in which those numbers control the timing of jumps.  At the
end of the game, the score is reported to the server and the server.

A slightly more directed baseline will have a server that starts with
a very low threshold for jumping that increases for subsequent games
based on the magnitude of the score reported by previous games.

\section{Techniques}
We will use reinforcement learning to evolve the behavior of our
player over a series of games.  In support of this task, we will
need to select the gameplay behavior features that make up our model
and develop a communication model between a learning process and
testing system that run in separate environments.

\section{Timeline}
We plan to set up our first baseline and ``learning'' communication
framework by March 10th and the second simple learning baseline by
March 31st.  This gives us ample time before the baseline results
deadline of April 14th to run tests and analyze the results.

\end{document}
